% *******************************************************
%		ANTECEDENTES PROFESIONALES
% *******************************************************

\spacedlowsmallcaps{C - Antecedentes de Investigaci\'on}
\vspace{1.5em}

\NewWorkExperience{2011-2012}{Detecci\'on y Extracci\'on de Sistemas orientados a Objetos,}{UNQ} \Description{
\MarginDate{Proyectos de Investigaci\'on \& Independientes}\newline
Durante los procesos de reingenier\'ia  de sistemas, es crucial  la detecci\'on de dependencias 
impl\'icitas entre los componentes de software de las aplicaciones. El ingeniero  de software
 debe identificar  estas dependencias para evitar afectar funcionalidad existente o agregar
 dependencias nuevas (e inapropiadas) cuando extienda o modifique el sistema.
 El An\'alisis  de Conceptos Formales (ACF) es una t\'ecnica  \'util cuando el ingeniero
 de software tiene el primer contacto con el sistema. Puede mostrar dependencias inesperadas y
 ayudar al desarrollador a entender qu\'e limitaciones  se imponen  en el sistema. Basado en la experiencia
 ganada durante los \'ultimos 10 a\~nos donde hemos aplicado el ACF en la reingenier\'ia de aplicaciones
 orientadas a objetos, este proyecto profundizar\'a  el an\'alisis de software basado en el ACF, con los
 siguientes objetivos: (1) Mejoras del an\'alisis basado en esta t\'ecnica, (2) el uso de t\'ecnicas 
 complementarias, como las m\'etricas y la informaci\'on din\'amica para generar la validaci\'on de mejores
  resultados, y (3) definici\'on de puntos en com\'un con t\'ecnicas existentes, como otros algoritmos de
  clustering.
  \vspace{0.5em} \newline
\textit{Referencias}: \href{mailto:gabriela.b.arevalo@gmail.com}{Dra. Gabriela Ar\'evalo} 
}


\vspace{1em}
\NewWorkExperience{2010}{Metamodelos para generar UI's,}{UNSAM} \Description{
Exploramos la idea de definir un metamodelo que nos permita inferir y generar la
interfaz de usuario de aplicaci\'on. Para esto desarrollamos este metamodelo y
tambi\'en identificamos y modelamos el dominio de la UI a trav\'es del patron
MMVC y el concepto de ``modelo de aplicaci\'on''. El objetivo es minimizar el
esfuerzo repetitivo y maximizar la consistencia en la construcci\'on de UI's. As\'i como tambi\'en proveer un marco evolutivo para
el desarrollo de aplicaciones. \newline
Implementamos dicha teor\'ia mediante la creaci\'on de un framework Java basado
en annotations como componentes declarativos del metamodelo.
\newline
\textit{Referencias}: \href{mailto:npasserini@gmail.com}{Ing. Nicol\'as Passerini} 
} 

\vspace{1em}

\NewWorkExperience{2010}{\href{https://sourceforge.net/projects/uqbar-commons/files/papers/object-transactions.pdf/download}{Object
Transactions}}{}

\Description{
Paper de investigaci\'on de un modelo de transacciones directamente asociado a
los conceptos principales del paradigma de objetos, desacoplado de los detalles
de implementaci\'on de los mecanismos de persistencia como BD relacionales.
}
\vspace{1em}

\NewWorkExperience{}{XCollections}{}

\Description{
Elaboraci\'on de un modelo de colecciones de alto nivel definidas por
comprensi\'on. Modela un ambiente o imagen del dominio, y permite alcanzar un alto grado de abstracci\'on
definiendo las colecciones como relaciones entre los objetos en lugar de meros
contenedores. Aplicaci\'on pr\'actica mediante la construcci\'on de un
framework de colecciones en Java.}

\vspace{1.5em}
\NewWorkExperience{}{BTTF}{}

\Description{
BTTF eleva el nivel de abstracci \'on del manejo del ``tiempo'' en las
aplicaciones. Permite modelar el dominio en t\'erminos de unidades (minutos,
meses, etc), per\'iodos, instantes, etc. Evitando el manejo de tipos
primitivos y fomentando el modelo de objetos.
}

\vspace{1.5em}
\NewWorkExperience{}{\href{http://sourceforge.net/projects/uqbar-commons/}{Proyecto
Uqbar}}{}

\Description{
Espacio com\'un para la contribuci\'on de proyectos de investigaci\'on y/o
herramientas \'utiles. Ejemplos: generaci\'on autom\'atica de diagramas de secuencia a partir
de la ejecuci\'on del sistema; integraci\'on del lenguaje PIC con el IDE eclipse; plugins de Maven
para automatizacion de tareas de SCM, etc.
}
